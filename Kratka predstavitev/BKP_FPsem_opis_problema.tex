\documentclass[a4paper, 11pt]{article}
\usepackage[slovene]{babel}
\usepackage[utf8]{inputenc}
\usepackage{graphicx}
\usepackage{fancyhdr}
\usepackage{biblatex}
\usepackage{enumerate}
\usepackage[normalem]{ulem}
\usepackage{amsmath,amsthm,amssymb}
\newtheorem{example}{Primer}
\usepackage{graphicx}
\usepackage{tabularx}
\usepackage{amsfonts}
\usepackage{eurosym}
\usepackage{tikz}
\usetikzlibrary{fadings,positioning,calc}
\tikzset{
	casual/.style={
	rectangle,
	draw=black, very thick,
	text width=6.5em,
	minimum height=2em,
	text centered
	}}
\newtheorem{theorem}{Trditev}
\newtheorem{definition}{Definicija}
\newtheorem{corollary}{Primer}


	\begin{document}
		\begin{titlepage}
			\begin{center}

			\large
			Univerza v Ljubljani\\
				\normalsize
				Fakulteta za matematiko in fiziko\\

			\vspace{5 cm} 

				\large
				Finančni praktikum\\


			\vspace{0.5cm}
			\LARGE
				\textbf{Algoritem za reševenje dvostopenjskega problema nahrbtnika z dinamičnim programiranjem}

			\vspace{0.5 cm}

			\large
				Jakob Zarnik, Tin Markon \\


			\vspace{1.5cm}
			\normalsize
				Mentorja: prof. dr. Sergia Cabello Justo, asist. dr. Janoš Vidali
			\vspace{3cm}


			\vfill

				\large Ljubljana, 2020

			\end{center}
		\end{titlepage}


\newpage

\tableofcontents
\vspace{22mm}

\newpage
		
	\pagestyle{fancy}
	\fancyhead{}
	\fancyhead[R]{Dvostopenjski problem nahrbtnika}
	\fancyfoot{}
	\fancyfoot[R]{\thepage}
	\fancyfoot[L]{Jakob Zarnik, Tin Markon}
	
	\begin{abstract}
	
	V nalogi bova obravnavala reševanje dvostopenjskega problema nahrbtnika z dinamičnim programiranjem. Za izdelavo algoritma bova uporabila programski jezik Python.
	
	\end{abstract}
	
	\pagebreak
	
	\section{Uvod}
	
	Dvostopenjski programi omogočajo modeliranje situacij, kjer glavni odločevalec, v nadaljevanju poimenovan investitor, optimizira svoja sredstva s tem, da neposredno upošteva odziv posrednika na njegovo odločitev o višini vložka. V primeru dvostopenjskega problema nahrbtnika (Bilevel Knapsack Problem), v nadaljevanju BKP, investitor določi prostornino nahrbtnika z namenom maksimizacije dobička, med tem ko se posrednik sooča z 0-1 problemom nahrbtnika s prostornino določeno s strani investitorja. BKP je ustrezen za modeliranje problema »ustreznega financiranja«, kjer posameznik (tj. investitor), svoja sredstva razdeli med netvegano naložbo s fiksnim donosom (npr. varčevalni račun, državna obveznica) in bolj tvegano naložbo, preko posrednika kot je banka ali bančni posrednik (broker). Ta kupi delnice ali obveznice z namenom maksimizacije svojega dobička s tem, da upošteva omejitve finančnih sredstev (prostornina nahrbtnika) investitorja in ustvari donos z ustrezno izbiro investicij. Podobno uporabo modeliranja lahko opazimo na področju upravljanja s proizvodi, kjer se podjetje odloča koliko enot izdelkov naj proda samo in koliko preko posrednika.
	
	BKP je mešan celoštevilski dvostopenjski problem predstavljen s strani Dempe in Richter, ki sta za rešitev predstavila "branch-and-bound" okvir (tj. razveji in omeji). V najini nalogi najprej razširiva potrebne in zadostne pogoje za obstoj optimalne rešitve. Nato predlagava enostaven in učinkovit algoritem dinamičnega programiranja za reševanje problema. V nasprotju s pristopom Dempe in Richter, kjer je beležen seznam nedominantnih rešitev, tukaj beležimo samo objektivne funkcijske vrednosti za oba, investitorja in posrednika, tekom dinamičnega procesa.
		
	\section{Formulacija in lastnosti problema}
	
	V narbtnik s prostornino oz kapaciteto $y$, ki jo določi investitor, vsakemu predmetu $j$ določimo utež oz. volumen $a_j$, zaslužek posrednika $c_j$ in zaslužek investitorja $d_j$. Ceno enote prostornine nahrbtnika označimo s $t$. Z danim $y$, posrednik izbere podmnožico predmetov, ki upošteva prostosrsko omejitev. To nam da dvostopenjski program
	\[   
	\text{BKP} = 
    	\begin{cases}
	 	\text{$\underset{y,x}{\text{Max}} f^{1}(y,x)=dx+ty$} \\
       		\text{s.t. $\underline{b} \leq y \leq \overline{b}$} \\
       		\text{$\underset{x}{\text{Max}} f^{2}(x)=cx$} \\
       		\text{s.t. $ax \leq y$ and $x \in \{ 0, 1\}^n$} \\ 
    	\end{cases}
	\]

	
kjer so $a$, $c$ in $d$ celoštevilske vrednosti in $a$, $c$, $d$, $\underline{b}$ in $\overline{b}$ nenegativne. V najini nalogi so z
	
	$$ \text{S} = \{ (x, y) \in \{ 0, 1 \}^n \times \left[ \underline{b}, \overline{b} \right]: ax \leq y\} $$ 
	
	označene omejitve, s 
	
	$$ \text{P}(y) = \{ x \in \text{Arg}\max \{ cx^\prime: ax^\prime \leq y, x^\prime \in \{ 0,1\}^n \} \}$$
	
	označimo posrednikovo racionalno izbiro množice (za fiksen $y$) in z 
	
	$$ \text{IR} = \{ (x,y)  | (x,y) \in S, x \in \text{P}(y) \} $$
	
	induktiven del, preko katerega investitor optimizira svojo funkcijo.
	
	Dvostopenjski program obstaja v dveh različicah. Optimističen primer, ko racionalna množica ni singelton (enolična), posrednik izbere tisto rešitev, ki maksimizira zaslužek investitorja. Dobljena rešitev se imenuje močna rešitev. V pesimističnem primeru pa investitor predvideva, da kadar ima posrednik več enakovrednih možnosti izbire množice, izbere tisto, ki minimizira investitorjev zaslužek. Tako dobimo šibko rešitev.
	
	\begin{theorem}[Dempe in Richter]
	Če je cena enote prostornine (enota $t$) nepozitivna, potem obstaja optimalna rešitev BKP.
	\label{trditev1}
	\end{theorem}
	
	Naslednja trditev povezje ceno prostornine z investitorjevim razmerjem med zaslužkom in utežjo predmeta.
	
	\begin{theorem}
	Naj bo $\underline{b} = 0$ in $t < 0$. Če je  $| t | > \underset{1 \leq j \leq n}{max}(\frac{d_j}{a_j})$,  potem je $(y^*, x^*) = (0, 0_n)$ optimalna rešitev.
	\end{theorem}
	
	\begin{proof}
	Naj bo $(x, y)$ možna rešitev za dan BKP. Najprej z razširitvijo pogoja $ ax \leq y$ s $t$ $(t < 0)$ dobimo $(ta + d)x \geq ty + dx = f^{1}(y,x)$. Nato, ker je $ta_j + d_j < 0$ za $j = 1, \dots , n$ in $ x \in \{0, 1 \}^n$, sledi, da je $(ta + d)x \leq 0$, torej $f^{1}(y,x) \leq 0$. Vidimo, da ker je $(y^*, x^*) = (0, 0_n)$ možna rešitev danega BKP, v katerem je $f^{1}(y,x) = 0$, je ta rešitev tudi optimalna.
	\end{proof}
	
	Če je $\infty > \overline{b} \geq \sum_{i=1}^na_i$ in $t > 0$, potem je optimalna rešitev trivialna: $x^* = (1, \dots, 1)$ in $y^* = \overline{b}$. Če sta $d$ in $c$ kolinearna ($d = \alpha c$, kjer $\alpha > 0$) in $t \geq 0$, potem je reševanje BKP enako reševanju problema nahrbtnika s kapaciteto $\overline{b}$ za posrednika.
	
	\begin{definition}
	Diskreten dvostopenjski problem nahrbtnika (BKPd) je dvostopenjski problem nahrbtnika v katerem je spremenljivka, ki jo določi investitor diskretna.
	\end{definition}

	\begin{theorem}
	Če je $t \leq 0$, potem je vsaka optimalna rešitev $(y^*, x^*)$ za BKPd, tudi optimalna rešitev za BKP.
	
	Če je $t > 0$ in če optimalna rešitev za BKP obstaja, potem je optimalna tudi za BKPd.	
	\end{theorem}
	
	\begin{proof}
	$t \leq 0$: Iz \textbf{Trditve 1} sledi, da optimalna rešitev $(y^*, x^*)$ obstaja. Dodatno, IR (BKPd) $\subset$ IR (BKP) in iz Dempe in Richter sledi, da je $y^*$ celo število.
	
	$t > 0$: Direktno iz $(ii)$ v Dempe in Richter.
	\end{proof}

	Iz \textbf{Trditve 3} sledi, da je reševanje BKP ekvivalentno reševanju BKPd, ko je $t$ negativen. Če je $t$ pozitiven in optimalna rešitev obstaja (glej Izrek 4 v Dempe in Richter), je ta dosežena v točki $(\overline{b}, x^*)$, kjer je $x^* \in P(\overline{b})$. Pomni, da optimalna rešitev BKPd vedno obstaja. Torej, če BKP ima optimalno rešitev, to lako dobimo z reševanjem zaporedja problemov nahrbtnika, ki vsebuje binarne spremenljivke, eno za vsak možno vrednost $y$. Algoritem, opisan v naslednjem poglavju, uporabi to lastnost.
	\section{Načrt za nadalnje delo}
	V nadaljevanju bova s programskim jezikom Python napisala program, ki s pomočjo dinamičnega programiranja izračuna optimalno rešitev $(y^*, x^*)$ glede na naključno generirane podatke. S pomočjo primerov iz dokumenta bova preverila tudi da program deluje pravilno. S pomočjo algoritma lahko izračunamo rešitev v primeru optimističnega primera, kot tudi pesimističnega. V najslabšem scenariju ima časovno zahtevnost $\theta(n\overline{b})$. Iz \textbf{Trditve 3} je razvidno, da je reševanje problema BKP ekvivalentno reševanju posrednikovega problema nahrbtnika za vsako celo število iz intervala $[ \underline{b}, \overline{b}]$. Algoritem v svojem teku jemlje obe ciljni funkciji (ali sta pri nama poimenovani drugače??). To dosežemo z dvema fazama, ki sta podrobneje opisani spodaj. \\

	\textbf{Forward phase} \\
	Prva faza je sestavljena iz dveh zank in nam zgenerira za vsak naslednji predmet $k$, ki ga dodamo v torbo in za vsako kapaciteto $y$. Opomniti je treba, da funkcija  $\widetilde{f}^{1}_{k}(y)$ ne upošteva ceno prostornine nahrbtnika, in je torej   $\widetilde{f}^{1}_{k}(y) = f^{1}_{k}(y)+ty$ \\

	\textbf{Backtracking phase} \\
	Druga faza je uporabljena za iskanje optimalne rešitve $(y^*, x^*)$, ki ustreza optimalni vrednosti določeni iz prejšnje faze. Optimalna kapaciteta $y$ je generirana z n-tim stolpcem tabele investitiorja, ki jo dobimo na sledeč način. 

	\begin{theorem} Naj bo $(y^*, x^*)$ optimalna rešitev za naš BKPd (Ali boma to sploh pisala tu?)
	
	Če je $t \leq 0$ potem je $\widetilde{f}^{1}_{n}(y^*) + ty^* = \max\{\widetilde{f}^{1}_{n}(y) + ty: y \in \{\underline{b}, \underline{b} + 1, \dots, \overline{b} \} \} $

	Če  je $t > 0$ potem obstajata dve možnosti:
	

	\end{theorem}
	
	
		\pagebreak


		
% VIRI	
		\begin{thebibliography}{9}
		\addcontentsline{toc}{section}{Literatura}
		
			\bibitem{Brotcorne,Hanafi,Mansi2009} L. Brotcorne, S. Hanafi, R. Mansi (2009).
			\textit{A dynamic programming algorithm for the bilevel knapsack problem} Elsevier: Operations Research Letters 37, 215–218.
			
    			\bibitem{Dempe,Richter2000} S. Dempe, K. Richter (2000). 
			\textit{Bilevel programming with Knapsack constraint} European Newspaper of Operations Research 8, 93–107.
			
		\end{thebibliography}
	
	
	
	
\end{document}
